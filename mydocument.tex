\documentclass{article}

\usepackage{url}
\usepackage{hyperref}
\usepackage{amsmath}
\usepackage{amsfonts}
\usepackage{physics}
\usepackage[barr,pdf]{xy}
\usepackage[numbered,framed]{matlab-prettifier}
\usepackage{comment}

\usepackage{amsthm}
\usepackage[capitalize]{cleveref}

\newtheorem{theorem}{Theorem}[section]
\newtheorem{exercise}[theorem]{Exercise}

\title{This is your title}
\author{Your Name}
\date{\today}

\begin{document}

\maketitle

\tableofcontents

\section{Linear maps}

A linear map $T$ transform elements of linear spaces guaranteeing
the following:

\begin{equation}
  \label{eq:linear-map}
  T(\alpha x + \beta y) = \alpha T(x) + \beta T(y).
\end{equation}

%% TODO: add an exercise in R^1 here.

It seems simple, and it is, yet that simplicity forms the
basis of most scientific computing, and a large amount of
general and business computing as well. Dig deep enough into
any sufficiently useful software, there are vector space
operations lurking somewhere.

To each linear map $T$ we may associate a matrix $\mathcal{M}(T)$;
your Daddy's matrix math follows.


\section{Bases}

We now return to coordinates, ordered tuples of which define vectors.
A vector may have 1, 2 or infinitely many coordinates, but we will confine
our investigation to vectors of finite (e.g., $n$) numbers of coordinates.
As a recap, and because it hasn't yet been made explicit, a brief review
of vector addition and scalar multiplication is in order.

\paragraph{Scalar multiplication} For some scalar $a$ in the field
over which the vector space is constructed, for some $v \in V$,
$av\in V$. For example, suppose $a = 2$ and $v' = [3, 4]$. Then
$av' = [6, 8]$.

\marginpar{We can't completely avoid your daddy's matrix math.}

\paragraph{Vector addition} By defintion, for vector $x, y \in V$,
$w = x + y \in V$. For example, take $F\equiv\mathbb{R}$. For
$x = [2, 3]'$ and $y = [4, 1]'$,

\begin{align}
  w &= \begin{bmatrix}6\\
             4\end{bmatrix} =
\begin{bmatrix}2\\
             3\end{bmatrix} +
\begin{bmatrix}4\\
             1\end{bmatrix}.
\end{align}

We're now equipped to investigate the fundamental operation
of a linear algebra, the linear combination.

\subsection{Linear combinations}

Suppose a linear space $V$ with elements $v, w \in V$ over some field
with members $a, b, c$, etc.
By definition~\ref{def:linear-space}, vectors may be composed of
\textit{linear combinations} of other vectors:

\begin{equation}
  w = a_1v_1 + \cdots + a_nv_n.
\end{equation}

\begin{comment}
  Primary concern here is establishing the equivalence between
  the number of linearly independent basis vectors and the dimension
  of the vector space.
\end{comment}

\begin{comment}
  From Wikipedia on Operads:
  \begin{quote}
    This point of view formalizes the
    notion that linear combinations are the most general sort of operation on a
    vector space – saying that a vector space is an algebra over the operad of
    linear combinations is precisely the statement that all possible algebraic
    operations in a vector space are linear combinations. The basic operations
    of vector addition and scalar multiplication are a generating set for the
    operad of all linear combinations, while the linear combinations operad
    canonically encodes all possible operations on a vector space.
  \end{quote}

\end{comment}

\subsection{Linear independence}

Suppose we have some linear combination of $v_1, v_2,\ldots, v_n$
for $i = 1..n, v_i \neq 0$ such that

\begin{equation}
  \label{eq:linear-independence}
  \alpha v_1 + \beta v_2 + \cdots + \omega v_n = 0.
\end{equation}

When Eq.~\ref{eq:linear-independence} implies $\alpha = \beta = \cdots = \omega = 0$, then
the vectors $v_1, v_2,\ldots, v_n$ are \textit{linearly independent}, by definition.
This means none of these vectors may be written as a linear combination of
the other vectors. Conversely, suppose one of the coefficients, e. g., $\alpha \neq 0$.
Then we can rewrite $v_1$ in terms of $v_2,\ldots, v_n$ as follows:

\[ % begin{equation}
  \alpha v_1 = -\beta v_2 - \cdots - \omega v_n,
\] % end{equation}
or
\begin{equation}
  \label{eq:linear-dependence}
  v_1 = \frac{-\beta}{\alpha}v_2 + \cdots + \frac{-\omega}{\alpha}v_n.
\end{equation}
From Eq.~\ref{eq:linear-dependence}, we say that the vectors $v_1, v_2,\ldots, v_n$
are \textit{linearly dependent}.

Linear independence leads naturally to the notion of \textit{span}, as discussed next.


\appendix

\section{Notation}

Notation is important as done well it helps increase understanding and
facilitate communication, while done poorly promotes confusion and
misunderstanding.  Notation for linear or vector space elements and operations
follows a range of conventions. After some exposure, notation ceases to be a
stumbling block, but it can be confusing for newcomers to the literature.

\subsection{bra--ket notation}

Dirac's ``bra--ket'' notation is standard in quantum mechanics, and looks
more intimidating than it really is. When the notation was developed, had
the distinct advantage of being very easy to type (on an actual, manual
typewriter), while being unambiguous.

\begin{align}
  \ket{x} &= \begin{bmatrix}
   x_{1} \\
   x_{2} \\
    \vdots \\
    x_{m}
  \end{bmatrix},
\end{align}

and
\begin{equation}
  \bra{x} = x^* = [x_1^*, x_2^*, \ldots, x_m^*].
\end{equation}

The notation $y^*$ indicates the hermitian conjugate; for our
purposes this is simply the complex conjugate of each coordinate
of $y$.

We care about this because we are learn it alls and this notation
appears in the quantum computing literature.

\begin{quote}
  In mathematical theories the question of notation, while not of primary
  importance, is yet worthy of careful consideration, since a good notation can
  be of great value in helping the development of a theory, by making it easy
  to write down those quantities or combinations of quantities that are
  important, and difficult or impossible to write down those that are
  unimportant. The summation convention in tensor analysis is an example,
  illustrating how specially appropriate a notation can be.\cite{dirac_1939}
\end{quote}

Very nice indeed: Dirac threw out the matrices.


\vspace{0.5in}

\hrule

$$
\bfig
\morphism[A`B;f]
\Loop(0,0)A(ur,ul)_g
\Loop(500,0)B(dl,dr)_h
\efig
$$

%\bibliography{math,text,dirac}{}
\bibliography{dirac}{}
\bibliographystyle{plain}

\end{document}
